\documentclass[11pt]{article}

    \usepackage[breakable]{tcolorbox}
    \usepackage{parskip} % Stop auto-indenting (to mimic markdown behaviour)
    

    % Basic figure setup, for now with no caption control since it's done
    % automatically by Pandoc (which extracts ![](path) syntax from Markdown).
    \usepackage{graphicx}
    % Maintain compatibility with old templates. Remove in nbconvert 6.0
    \let\Oldincludegraphics\includegraphics
    % Ensure that by default, figures have no caption (until we provide a
    % proper Figure object with a Caption API and a way to capture that
    % in the conversion process - todo).
    \usepackage{caption}
    \DeclareCaptionFormat{nocaption}{}
    \captionsetup{format=nocaption,aboveskip=0pt,belowskip=0pt}

    \usepackage{float}
    \floatplacement{figure}{H} % forces figures to be placed at the correct location
    \usepackage{xcolor} % Allow colors to be defined
    \usepackage{enumerate} % Needed for markdown enumerations to work
    \usepackage{geometry} % Used to adjust the document margins
    \usepackage{amsmath} % Equations
    \usepackage{amssymb} % Equations
    \usepackage{textcomp} % defines textquotesingle
    % Hack from http://tex.stackexchange.com/a/47451/13684:
    \AtBeginDocument{%
        \def\PYZsq{\textquotesingle}% Upright quotes in Pygmentized code
    }
    \usepackage{upquote} % Upright quotes for verbatim code
    \usepackage{eurosym} % defines \euro

    \usepackage{iftex}
    \ifPDFTeX
        \usepackage[T1]{fontenc}
        \IfFileExists{alphabeta.sty}{
              \usepackage{alphabeta}
          }{
              \usepackage[mathletters]{ucs}
              \usepackage[utf8x]{inputenc}
          }
    \else
        \usepackage{fontspec}
        \usepackage{unicode-math}
    \fi

    \usepackage{fancyvrb} % verbatim replacement that allows latex
    \usepackage[Export]{adjustbox} % Used to constrain images to a maximum size
    \adjustboxset{max size={0.9\linewidth}{0.9\paperheight}}

    % The hyperref package gives us a pdf with properly built
    % internal navigation ('pdf bookmarks' for the table of contents,
    % internal cross-reference links, web links for URLs, etc.)
    \usepackage{hyperref}
    % The default LaTeX title has an obnoxious amount of whitespace. By default,
    % titling removes some of it. It also provides customization options.
    \usepackage{titling}
    \usepackage{longtable} % longtable support required by pandoc >1.10
    \usepackage{booktabs}  % table support for pandoc > 1.12.2
    \usepackage{array}     % table support for pandoc >= 2.11.3
    \usepackage{calc}      % table minipage width calculation for pandoc >= 2.11.1
    \usepackage[inline]{enumitem} % IRkernel/repr support (it uses the enumerate* environment)
    \usepackage[normalem]{ulem} % ulem is needed to support strikethroughs (\sout)
                                % normalem makes italics be italics, not underlines
    \usepackage{mathrsfs}
    

    
    % Colors for the hyperref package
    \definecolor{urlcolor}{rgb}{0,.145,.698}
    \definecolor{linkcolor}{rgb}{.71,0.21,0.01}
    \definecolor{citecolor}{rgb}{.12,.54,.11}

    % ANSI colors
    \definecolor{ansi-black}{HTML}{3E424D}
    \definecolor{ansi-black-intense}{HTML}{282C36}
    \definecolor{ansi-red}{HTML}{E75C58}
    \definecolor{ansi-red-intense}{HTML}{B22B31}
    \definecolor{ansi-green}{HTML}{00A250}
    \definecolor{ansi-green-intense}{HTML}{007427}
    \definecolor{ansi-yellow}{HTML}{DDB62B}
    \definecolor{ansi-yellow-intense}{HTML}{B27D12}
    \definecolor{ansi-blue}{HTML}{208FFB}
    \definecolor{ansi-blue-intense}{HTML}{0065CA}
    \definecolor{ansi-magenta}{HTML}{D160C4}
    \definecolor{ansi-magenta-intense}{HTML}{A03196}
    \definecolor{ansi-cyan}{HTML}{60C6C8}
    \definecolor{ansi-cyan-intense}{HTML}{258F8F}
    \definecolor{ansi-white}{HTML}{C5C1B4}
    \definecolor{ansi-white-intense}{HTML}{A1A6B2}
    \definecolor{ansi-default-inverse-fg}{HTML}{FFFFFF}
    \definecolor{ansi-default-inverse-bg}{HTML}{000000}

    % common color for the border for error outputs.
    \definecolor{outerrorbackground}{HTML}{FFDFDF}

    % commands and environments needed by pandoc snippets
    % extracted from the output of `pandoc -s`
    \providecommand{\tightlist}{%
      \setlength{\itemsep}{0pt}\setlength{\parskip}{0pt}}
    \DefineVerbatimEnvironment{Highlighting}{Verbatim}{commandchars=\\\{\}}
    % Add ',fontsize=\small' for more characters per line
    \newenvironment{Shaded}{}{}
    \newcommand{\KeywordTok}[1]{\textcolor[rgb]{0.00,0.44,0.13}{\textbf{{#1}}}}
    \newcommand{\DataTypeTok}[1]{\textcolor[rgb]{0.56,0.13,0.00}{{#1}}}
    \newcommand{\DecValTok}[1]{\textcolor[rgb]{0.25,0.63,0.44}{{#1}}}
    \newcommand{\BaseNTok}[1]{\textcolor[rgb]{0.25,0.63,0.44}{{#1}}}
    \newcommand{\FloatTok}[1]{\textcolor[rgb]{0.25,0.63,0.44}{{#1}}}
    \newcommand{\CharTok}[1]{\textcolor[rgb]{0.25,0.44,0.63}{{#1}}}
    \newcommand{\StringTok}[1]{\textcolor[rgb]{0.25,0.44,0.63}{{#1}}}
    \newcommand{\CommentTok}[1]{\textcolor[rgb]{0.38,0.63,0.69}{\textit{{#1}}}}
    \newcommand{\OtherTok}[1]{\textcolor[rgb]{0.00,0.44,0.13}{{#1}}}
    \newcommand{\AlertTok}[1]{\textcolor[rgb]{1.00,0.00,0.00}{\textbf{{#1}}}}
    \newcommand{\FunctionTok}[1]{\textcolor[rgb]{0.02,0.16,0.49}{{#1}}}
    \newcommand{\RegionMarkerTok}[1]{{#1}}
    \newcommand{\ErrorTok}[1]{\textcolor[rgb]{1.00,0.00,0.00}{\textbf{{#1}}}}
    \newcommand{\NormalTok}[1]{{#1}}
    
    % Additional commands for more recent versions of Pandoc
    \newcommand{\ConstantTok}[1]{\textcolor[rgb]{0.53,0.00,0.00}{{#1}}}
    \newcommand{\SpecialCharTok}[1]{\textcolor[rgb]{0.25,0.44,0.63}{{#1}}}
    \newcommand{\VerbatimStringTok}[1]{\textcolor[rgb]{0.25,0.44,0.63}{{#1}}}
    \newcommand{\SpecialStringTok}[1]{\textcolor[rgb]{0.73,0.40,0.53}{{#1}}}
    \newcommand{\ImportTok}[1]{{#1}}
    \newcommand{\DocumentationTok}[1]{\textcolor[rgb]{0.73,0.13,0.13}{\textit{{#1}}}}
    \newcommand{\AnnotationTok}[1]{\textcolor[rgb]{0.38,0.63,0.69}{\textbf{\textit{{#1}}}}}
    \newcommand{\CommentVarTok}[1]{\textcolor[rgb]{0.38,0.63,0.69}{\textbf{\textit{{#1}}}}}
    \newcommand{\VariableTok}[1]{\textcolor[rgb]{0.10,0.09,0.49}{{#1}}}
    \newcommand{\ControlFlowTok}[1]{\textcolor[rgb]{0.00,0.44,0.13}{\textbf{{#1}}}}
    \newcommand{\OperatorTok}[1]{\textcolor[rgb]{0.40,0.40,0.40}{{#1}}}
    \newcommand{\BuiltInTok}[1]{{#1}}
    \newcommand{\ExtensionTok}[1]{{#1}}
    \newcommand{\PreprocessorTok}[1]{\textcolor[rgb]{0.74,0.48,0.00}{{#1}}}
    \newcommand{\AttributeTok}[1]{\textcolor[rgb]{0.49,0.56,0.16}{{#1}}}
    \newcommand{\InformationTok}[1]{\textcolor[rgb]{0.38,0.63,0.69}{\textbf{\textit{{#1}}}}}
    \newcommand{\WarningTok}[1]{\textcolor[rgb]{0.38,0.63,0.69}{\textbf{\textit{{#1}}}}}
    
    
    % Define a nice break command that doesn't care if a line doesn't already
    % exist.
    \def\br{\hspace*{\fill} \\* }
    % Math Jax compatibility definitions
    \def\gt{>}
    \def\lt{<}
    \let\Oldtex\TeX
    \let\Oldlatex\LaTeX
    \renewcommand{\TeX}{\textrm{\Oldtex}}
    \renewcommand{\LaTeX}{\textrm{\Oldlatex}}
    % Document parameters
    % Document title
    \title{Tarea\_2}
    
    
    
    
    
% Pygments definitions
\makeatletter
\def\PY@reset{\let\PY@it=\relax \let\PY@bf=\relax%
    \let\PY@ul=\relax \let\PY@tc=\relax%
    \let\PY@bc=\relax \let\PY@ff=\relax}
\def\PY@tok#1{\csname PY@tok@#1\endcsname}
\def\PY@toks#1+{\ifx\relax#1\empty\else%
    \PY@tok{#1}\expandafter\PY@toks\fi}
\def\PY@do#1{\PY@bc{\PY@tc{\PY@ul{%
    \PY@it{\PY@bf{\PY@ff{#1}}}}}}}
\def\PY#1#2{\PY@reset\PY@toks#1+\relax+\PY@do{#2}}

\@namedef{PY@tok@w}{\def\PY@tc##1{\textcolor[rgb]{0.73,0.73,0.73}{##1}}}
\@namedef{PY@tok@c}{\let\PY@it=\textit\def\PY@tc##1{\textcolor[rgb]{0.24,0.48,0.48}{##1}}}
\@namedef{PY@tok@cp}{\def\PY@tc##1{\textcolor[rgb]{0.61,0.40,0.00}{##1}}}
\@namedef{PY@tok@k}{\let\PY@bf=\textbf\def\PY@tc##1{\textcolor[rgb]{0.00,0.50,0.00}{##1}}}
\@namedef{PY@tok@kp}{\def\PY@tc##1{\textcolor[rgb]{0.00,0.50,0.00}{##1}}}
\@namedef{PY@tok@kt}{\def\PY@tc##1{\textcolor[rgb]{0.69,0.00,0.25}{##1}}}
\@namedef{PY@tok@o}{\def\PY@tc##1{\textcolor[rgb]{0.40,0.40,0.40}{##1}}}
\@namedef{PY@tok@ow}{\let\PY@bf=\textbf\def\PY@tc##1{\textcolor[rgb]{0.67,0.13,1.00}{##1}}}
\@namedef{PY@tok@nb}{\def\PY@tc##1{\textcolor[rgb]{0.00,0.50,0.00}{##1}}}
\@namedef{PY@tok@nf}{\def\PY@tc##1{\textcolor[rgb]{0.00,0.00,1.00}{##1}}}
\@namedef{PY@tok@nc}{\let\PY@bf=\textbf\def\PY@tc##1{\textcolor[rgb]{0.00,0.00,1.00}{##1}}}
\@namedef{PY@tok@nn}{\let\PY@bf=\textbf\def\PY@tc##1{\textcolor[rgb]{0.00,0.00,1.00}{##1}}}
\@namedef{PY@tok@ne}{\let\PY@bf=\textbf\def\PY@tc##1{\textcolor[rgb]{0.80,0.25,0.22}{##1}}}
\@namedef{PY@tok@nv}{\def\PY@tc##1{\textcolor[rgb]{0.10,0.09,0.49}{##1}}}
\@namedef{PY@tok@no}{\def\PY@tc##1{\textcolor[rgb]{0.53,0.00,0.00}{##1}}}
\@namedef{PY@tok@nl}{\def\PY@tc##1{\textcolor[rgb]{0.46,0.46,0.00}{##1}}}
\@namedef{PY@tok@ni}{\let\PY@bf=\textbf\def\PY@tc##1{\textcolor[rgb]{0.44,0.44,0.44}{##1}}}
\@namedef{PY@tok@na}{\def\PY@tc##1{\textcolor[rgb]{0.41,0.47,0.13}{##1}}}
\@namedef{PY@tok@nt}{\let\PY@bf=\textbf\def\PY@tc##1{\textcolor[rgb]{0.00,0.50,0.00}{##1}}}
\@namedef{PY@tok@nd}{\def\PY@tc##1{\textcolor[rgb]{0.67,0.13,1.00}{##1}}}
\@namedef{PY@tok@s}{\def\PY@tc##1{\textcolor[rgb]{0.73,0.13,0.13}{##1}}}
\@namedef{PY@tok@sd}{\let\PY@it=\textit\def\PY@tc##1{\textcolor[rgb]{0.73,0.13,0.13}{##1}}}
\@namedef{PY@tok@si}{\let\PY@bf=\textbf\def\PY@tc##1{\textcolor[rgb]{0.64,0.35,0.47}{##1}}}
\@namedef{PY@tok@se}{\let\PY@bf=\textbf\def\PY@tc##1{\textcolor[rgb]{0.67,0.36,0.12}{##1}}}
\@namedef{PY@tok@sr}{\def\PY@tc##1{\textcolor[rgb]{0.64,0.35,0.47}{##1}}}
\@namedef{PY@tok@ss}{\def\PY@tc##1{\textcolor[rgb]{0.10,0.09,0.49}{##1}}}
\@namedef{PY@tok@sx}{\def\PY@tc##1{\textcolor[rgb]{0.00,0.50,0.00}{##1}}}
\@namedef{PY@tok@m}{\def\PY@tc##1{\textcolor[rgb]{0.40,0.40,0.40}{##1}}}
\@namedef{PY@tok@gh}{\let\PY@bf=\textbf\def\PY@tc##1{\textcolor[rgb]{0.00,0.00,0.50}{##1}}}
\@namedef{PY@tok@gu}{\let\PY@bf=\textbf\def\PY@tc##1{\textcolor[rgb]{0.50,0.00,0.50}{##1}}}
\@namedef{PY@tok@gd}{\def\PY@tc##1{\textcolor[rgb]{0.63,0.00,0.00}{##1}}}
\@namedef{PY@tok@gi}{\def\PY@tc##1{\textcolor[rgb]{0.00,0.52,0.00}{##1}}}
\@namedef{PY@tok@gr}{\def\PY@tc##1{\textcolor[rgb]{0.89,0.00,0.00}{##1}}}
\@namedef{PY@tok@ge}{\let\PY@it=\textit}
\@namedef{PY@tok@gs}{\let\PY@bf=\textbf}
\@namedef{PY@tok@gp}{\let\PY@bf=\textbf\def\PY@tc##1{\textcolor[rgb]{0.00,0.00,0.50}{##1}}}
\@namedef{PY@tok@go}{\def\PY@tc##1{\textcolor[rgb]{0.44,0.44,0.44}{##1}}}
\@namedef{PY@tok@gt}{\def\PY@tc##1{\textcolor[rgb]{0.00,0.27,0.87}{##1}}}
\@namedef{PY@tok@err}{\def\PY@bc##1{{\setlength{\fboxsep}{\string -\fboxrule}\fcolorbox[rgb]{1.00,0.00,0.00}{1,1,1}{\strut ##1}}}}
\@namedef{PY@tok@kc}{\let\PY@bf=\textbf\def\PY@tc##1{\textcolor[rgb]{0.00,0.50,0.00}{##1}}}
\@namedef{PY@tok@kd}{\let\PY@bf=\textbf\def\PY@tc##1{\textcolor[rgb]{0.00,0.50,0.00}{##1}}}
\@namedef{PY@tok@kn}{\let\PY@bf=\textbf\def\PY@tc##1{\textcolor[rgb]{0.00,0.50,0.00}{##1}}}
\@namedef{PY@tok@kr}{\let\PY@bf=\textbf\def\PY@tc##1{\textcolor[rgb]{0.00,0.50,0.00}{##1}}}
\@namedef{PY@tok@bp}{\def\PY@tc##1{\textcolor[rgb]{0.00,0.50,0.00}{##1}}}
\@namedef{PY@tok@fm}{\def\PY@tc##1{\textcolor[rgb]{0.00,0.00,1.00}{##1}}}
\@namedef{PY@tok@vc}{\def\PY@tc##1{\textcolor[rgb]{0.10,0.09,0.49}{##1}}}
\@namedef{PY@tok@vg}{\def\PY@tc##1{\textcolor[rgb]{0.10,0.09,0.49}{##1}}}
\@namedef{PY@tok@vi}{\def\PY@tc##1{\textcolor[rgb]{0.10,0.09,0.49}{##1}}}
\@namedef{PY@tok@vm}{\def\PY@tc##1{\textcolor[rgb]{0.10,0.09,0.49}{##1}}}
\@namedef{PY@tok@sa}{\def\PY@tc##1{\textcolor[rgb]{0.73,0.13,0.13}{##1}}}
\@namedef{PY@tok@sb}{\def\PY@tc##1{\textcolor[rgb]{0.73,0.13,0.13}{##1}}}
\@namedef{PY@tok@sc}{\def\PY@tc##1{\textcolor[rgb]{0.73,0.13,0.13}{##1}}}
\@namedef{PY@tok@dl}{\def\PY@tc##1{\textcolor[rgb]{0.73,0.13,0.13}{##1}}}
\@namedef{PY@tok@s2}{\def\PY@tc##1{\textcolor[rgb]{0.73,0.13,0.13}{##1}}}
\@namedef{PY@tok@sh}{\def\PY@tc##1{\textcolor[rgb]{0.73,0.13,0.13}{##1}}}
\@namedef{PY@tok@s1}{\def\PY@tc##1{\textcolor[rgb]{0.73,0.13,0.13}{##1}}}
\@namedef{PY@tok@mb}{\def\PY@tc##1{\textcolor[rgb]{0.40,0.40,0.40}{##1}}}
\@namedef{PY@tok@mf}{\def\PY@tc##1{\textcolor[rgb]{0.40,0.40,0.40}{##1}}}
\@namedef{PY@tok@mh}{\def\PY@tc##1{\textcolor[rgb]{0.40,0.40,0.40}{##1}}}
\@namedef{PY@tok@mi}{\def\PY@tc##1{\textcolor[rgb]{0.40,0.40,0.40}{##1}}}
\@namedef{PY@tok@il}{\def\PY@tc##1{\textcolor[rgb]{0.40,0.40,0.40}{##1}}}
\@namedef{PY@tok@mo}{\def\PY@tc##1{\textcolor[rgb]{0.40,0.40,0.40}{##1}}}
\@namedef{PY@tok@ch}{\let\PY@it=\textit\def\PY@tc##1{\textcolor[rgb]{0.24,0.48,0.48}{##1}}}
\@namedef{PY@tok@cm}{\let\PY@it=\textit\def\PY@tc##1{\textcolor[rgb]{0.24,0.48,0.48}{##1}}}
\@namedef{PY@tok@cpf}{\let\PY@it=\textit\def\PY@tc##1{\textcolor[rgb]{0.24,0.48,0.48}{##1}}}
\@namedef{PY@tok@c1}{\let\PY@it=\textit\def\PY@tc##1{\textcolor[rgb]{0.24,0.48,0.48}{##1}}}
\@namedef{PY@tok@cs}{\let\PY@it=\textit\def\PY@tc##1{\textcolor[rgb]{0.24,0.48,0.48}{##1}}}

\def\PYZbs{\char`\\}
\def\PYZus{\char`\_}
\def\PYZob{\char`\{}
\def\PYZcb{\char`\}}
\def\PYZca{\char`\^}
\def\PYZam{\char`\&}
\def\PYZlt{\char`\<}
\def\PYZgt{\char`\>}
\def\PYZsh{\char`\#}
\def\PYZpc{\char`\%}
\def\PYZdl{\char`\$}
\def\PYZhy{\char`\-}
\def\PYZsq{\char`\'}
\def\PYZdq{\char`\"}
\def\PYZti{\char`\~}
% for compatibility with earlier versions
\def\PYZat{@}
\def\PYZlb{[}
\def\PYZrb{]}
\makeatother


    % For linebreaks inside Verbatim environment from package fancyvrb. 
    \makeatletter
        \newbox\Wrappedcontinuationbox 
        \newbox\Wrappedvisiblespacebox 
        \newcommand*\Wrappedvisiblespace {\textcolor{red}{\textvisiblespace}} 
        \newcommand*\Wrappedcontinuationsymbol {\textcolor{red}{\llap{\tiny$\m@th\hookrightarrow$}}} 
        \newcommand*\Wrappedcontinuationindent {3ex } 
        \newcommand*\Wrappedafterbreak {\kern\Wrappedcontinuationindent\copy\Wrappedcontinuationbox} 
        % Take advantage of the already applied Pygments mark-up to insert 
        % potential linebreaks for TeX processing. 
        %        {, <, #, %, $, ' and ": go to next line. 
        %        _, }, ^, &, >, - and ~: stay at end of broken line. 
        % Use of \textquotesingle for straight quote. 
        \newcommand*\Wrappedbreaksatspecials {% 
            \def\PYGZus{\discretionary{\char`\_}{\Wrappedafterbreak}{\char`\_}}% 
            \def\PYGZob{\discretionary{}{\Wrappedafterbreak\char`\{}{\char`\{}}% 
            \def\PYGZcb{\discretionary{\char`\}}{\Wrappedafterbreak}{\char`\}}}% 
            \def\PYGZca{\discretionary{\char`\^}{\Wrappedafterbreak}{\char`\^}}% 
            \def\PYGZam{\discretionary{\char`\&}{\Wrappedafterbreak}{\char`\&}}% 
            \def\PYGZlt{\discretionary{}{\Wrappedafterbreak\char`\<}{\char`\<}}% 
            \def\PYGZgt{\discretionary{\char`\>}{\Wrappedafterbreak}{\char`\>}}% 
            \def\PYGZsh{\discretionary{}{\Wrappedafterbreak\char`\#}{\char`\#}}% 
            \def\PYGZpc{\discretionary{}{\Wrappedafterbreak\char`\%}{\char`\%}}% 
            \def\PYGZdl{\discretionary{}{\Wrappedafterbreak\char`\$}{\char`\$}}% 
            \def\PYGZhy{\discretionary{\char`\-}{\Wrappedafterbreak}{\char`\-}}% 
            \def\PYGZsq{\discretionary{}{\Wrappedafterbreak\textquotesingle}{\textquotesingle}}% 
            \def\PYGZdq{\discretionary{}{\Wrappedafterbreak\char`\"}{\char`\"}}% 
            \def\PYGZti{\discretionary{\char`\~}{\Wrappedafterbreak}{\char`\~}}% 
        } 
        % Some characters . , ; ? ! / are not pygmentized. 
        % This macro makes them "active" and they will insert potential linebreaks 
        \newcommand*\Wrappedbreaksatpunct {% 
            \lccode`\~`\.\lowercase{\def~}{\discretionary{\hbox{\char`\.}}{\Wrappedafterbreak}{\hbox{\char`\.}}}% 
            \lccode`\~`\,\lowercase{\def~}{\discretionary{\hbox{\char`\,}}{\Wrappedafterbreak}{\hbox{\char`\,}}}% 
            \lccode`\~`\;\lowercase{\def~}{\discretionary{\hbox{\char`\;}}{\Wrappedafterbreak}{\hbox{\char`\;}}}% 
            \lccode`\~`\:\lowercase{\def~}{\discretionary{\hbox{\char`\:}}{\Wrappedafterbreak}{\hbox{\char`\:}}}% 
            \lccode`\~`\?\lowercase{\def~}{\discretionary{\hbox{\char`\?}}{\Wrappedafterbreak}{\hbox{\char`\?}}}% 
            \lccode`\~`\!\lowercase{\def~}{\discretionary{\hbox{\char`\!}}{\Wrappedafterbreak}{\hbox{\char`\!}}}% 
            \lccode`\~`\/\lowercase{\def~}{\discretionary{\hbox{\char`\/}}{\Wrappedafterbreak}{\hbox{\char`\/}}}% 
            \catcode`\.\active
            \catcode`\,\active 
            \catcode`\;\active
            \catcode`\:\active
            \catcode`\?\active
            \catcode`\!\active
            \catcode`\/\active 
            \lccode`\~`\~ 	
        }
    \makeatother

    \let\OriginalVerbatim=\Verbatim
    \makeatletter
    \renewcommand{\Verbatim}[1][1]{%
        %\parskip\z@skip
        \sbox\Wrappedcontinuationbox {\Wrappedcontinuationsymbol}%
        \sbox\Wrappedvisiblespacebox {\FV@SetupFont\Wrappedvisiblespace}%
        \def\FancyVerbFormatLine ##1{\hsize\linewidth
            \vtop{\raggedright\hyphenpenalty\z@\exhyphenpenalty\z@
                \doublehyphendemerits\z@\finalhyphendemerits\z@
                \strut ##1\strut}%
        }%
        % If the linebreak is at a space, the latter will be displayed as visible
        % space at end of first line, and a continuation symbol starts next line.
        % Stretch/shrink are however usually zero for typewriter font.
        \def\FV@Space {%
            \nobreak\hskip\z@ plus\fontdimen3\font minus\fontdimen4\font
            \discretionary{\copy\Wrappedvisiblespacebox}{\Wrappedafterbreak}
            {\kern\fontdimen2\font}%
        }%
        
        % Allow breaks at special characters using \PYG... macros.
        \Wrappedbreaksatspecials
        % Breaks at punctuation characters . , ; ? ! and / need catcode=\active 	
        \OriginalVerbatim[#1,codes*=\Wrappedbreaksatpunct]%
    }
    \makeatother

    % Exact colors from NB
    \definecolor{incolor}{HTML}{303F9F}
    \definecolor{outcolor}{HTML}{D84315}
    \definecolor{cellborder}{HTML}{CFCFCF}
    \definecolor{cellbackground}{HTML}{F7F7F7}
    
    % prompt
    \makeatletter
    \newcommand{\boxspacing}{\kern\kvtcb@left@rule\kern\kvtcb@boxsep}
    \makeatother
    \newcommand{\prompt}[4]{
        {\ttfamily\llap{{\color{#2}[#3]:\hspace{3pt}#4}}\vspace{-\baselineskip}}
    }
    

    
    % Prevent overflowing lines due to hard-to-break entities
    \sloppy 
    % Setup hyperref package
    \hypersetup{
      breaklinks=true,  % so long urls are correctly broken across lines
      colorlinks=true,
      urlcolor=urlcolor,
      linkcolor=linkcolor,
      citecolor=citecolor,
      }
    % Slightly bigger margins than the latex defaults
    
    \geometry{verbose,tmargin=1in,bmargin=1in,lmargin=1in,rmargin=1in}
    
    

\begin{document}
    
    \maketitle
    
    

    
    \hypertarget{fundamentos-de-programaciuxf3n-para-la-fuxedsica-computacional}{%
\section{Fundamentos de programación para la física
computacional}\label{fundamentos-de-programaciuxf3n-para-la-fuxedsica-computacional}}

\hypertarget{luis-daniel-amador-islas}{%
\subsubsection{Luis Daniel Amador
Islas}\label{luis-daniel-amador-islas}}

    \hypertarget{constante-de-madelung}{%
\subsection{1. Constante de Madelung}\label{constante-de-madelung}}

En física de la materia condensada, la \emph{constante de Madelung} da
el potencial eléctrico total que siente un átomo en un sólido; y depende
de las cargas de los otros átomos cercanos y de sus ubicaciones.

Por ejemplo, el cristal de cloruro de sodio sólido (la sal de mesa),
tiene átomos dispuestos en una red cúbica, con átomos de sodio y cloro
alternados, teniendo los de sodio una carga posiiva \(+e\) y los de
cloro una negativa \(-e\),(donde \(e\) es la carga del electrón). Si
etiquetamos cada posición en la red con tres coordenadas enteras
(\(i,j,k\)), entonces los átomos de sodio caen en posiciones donde
\(i+j+k\) es par, y los átomos de cloro en psocion donde \(i+j+k\) es
impar. Consideremos un átomo de sodio en el origen, i.e.~\(i=j=k=0\), y
calculemos la \emph{constante de Madelung}. Si el espaciado de los
átomos en la red es \(a\), entonces la distancia desde el origen al
átomo en la posicion (i,j,k) es:

\[
\sqrt{(ia)^2 + (ja)^2 + (ka)^2} = a\sqrt{i^2 + j^2 + k^2}
\]

y el potencial en el origen creado por tal átomo es:

\[
V(i,j,k) = ±\frac{e}{4 \pi \epsilon_0a \sqrt{i^2 + j^2 + k^2}}
\]

siendo \(\epsilon_0\) la permitividad del vacío y el signo de la
expresión se toma dependiendo de si \(i +j +k\) es par o impar. Así
entonces, el potencial otal que siente el átomo de sodio es la suma de
esta cantidad sobre todos los demas átomos. Supongamos una caja cúbica
alrededor del átomo de sodio en el origen, con L átomos en todas las
direcciones, entonces:

\[
V_{total} = \sum_{i,j,k=-L} V(i,j,k) = \frac{e}{4\pi \epsilon_0 a} M,
\]

donde (M) es la constante de Madelung (al menos aproximadamente).

Técnicamente, la constante de Madelung es el valor de \(M\) cuando
\(L \to \infty\), pero se puede obener una buena aproximación
simplemente usando un valor grande de \(L\). Escribe un programa para
calcular e imprimir la \emph{constante de Madelung} para el cloruro de
sodio. Utiliza un valor de \(L\) tan grande como puedas, sin dejar que
tu programa se ejecutar en un tiempo raoznable (un minuto o menos).

    \hypertarget{respuesta}{%
\subsection{RESPUESTA}\label{respuesta}}

    \begin{tcolorbox}[breakable, size=fbox, boxrule=1pt, pad at break*=1mm,colback=cellbackground, colframe=cellborder]
\prompt{In}{incolor}{25}{\boxspacing}
\begin{Verbatim}[commandchars=\\\{\}]
\PY{k+kn}{import} \PY{n+nn}{numpy} \PY{k}{as} \PY{n+nn}{np}

\PY{k}{def} \PY{n+nf}{V}\PY{p}{(}\PY{n}{i}\PY{p}{,}\PY{n}{j}\PY{p}{,}\PY{n}{k}\PY{p}{)}\PY{p}{:}
    \PY{k}{return} \PY{p}{(}\PY{o}{\PYZhy{}}\PY{l+m+mi}{1}\PY{p}{)}\PY{o}{*}\PY{o}{*}\PY{p}{(}\PY{n}{i}\PY{o}{+}\PY{n}{j}\PY{o}{+}\PY{n}{k}\PY{p}{)} \PY{o}{/} \PY{n}{math}\PY{o}{.}\PY{n}{sqrt}\PY{p}{(} \PY{n}{i}\PY{o}{*}\PY{n}{i} \PY{o}{+} \PY{n}{j}\PY{o}{*}\PY{n}{j} \PY{o}{+} \PY{n}{k}\PY{o}{*}\PY{n}{k} \PY{p}{)}
    
\PY{n}{L} \PY{o}{=} \PY{l+m+mi}{100}
\PY{n}{V\PYZus{}total} \PY{o}{=} \PY{l+m+mf}{0.0} 
\PY{k}{for} \PY{n}{i} \PY{o+ow}{in} \PY{n+nb}{range}\PY{p}{(}\PY{o}{\PYZhy{}}\PY{n}{L}\PY{p}{,} \PY{n}{L} \PY{o}{+} \PY{l+m+mi}{1}\PY{p}{)}\PY{p}{:}
    \PY{k}{for} \PY{n}{j} \PY{o+ow}{in} \PY{n+nb}{range}\PY{p}{(}\PY{o}{\PYZhy{}}\PY{n}{L}\PY{p}{,} \PY{n}{L} \PY{o}{+} \PY{l+m+mi}{1}\PY{p}{)}\PY{p}{:}
        \PY{k}{for} \PY{n}{k} \PY{o+ow}{in} \PY{n+nb}{range}\PY{p}{(}\PY{o}{\PYZhy{}}\PY{n}{L}\PY{p}{,} \PY{n}{L} \PY{o}{+} \PY{l+m+mi}{1}\PY{p}{)}\PY{p}{:}
            \PY{k}{if} \PY{n}{i} \PY{o}{==} \PY{n}{j} \PY{o}{==} \PY{n}{k} \PY{o}{==} \PY{l+m+mi}{0}\PY{p}{:}
                \PY{k}{continue}
            \PY{k}{elif} \PY{p}{(}\PY{n}{i}\PY{o}{*}\PY{n}{j}\PY{o}{*}\PY{n}{k}\PY{p}{)}\PY{o}{*}\PY{l+m+mi}{2} \PY{o}{==} \PY{l+m+mi}{0}\PY{p}{:}
                \PY{n}{V\PYZus{}total} \PY{o}{+}\PY{o}{=} \PY{n}{V}\PY{p}{(}\PY{n}{i}\PY{p}{,}\PY{n}{j}\PY{p}{,}\PY{n}{k}\PY{p}{)}
            \PY{k}{else}\PY{p}{:}
                \PY{n}{V\PYZus{}total} \PY{o}{+}\PY{o}{=} \PY{n}{V}\PY{p}{(}\PY{n}{i}\PY{p}{,}\PY{n}{j}\PY{p}{,}\PY{n}{k}\PY{p}{)}

\PY{n+nb}{print}\PY{p}{(}\PY{n}{V\PYZus{}total}\PY{p}{)}
\end{Verbatim}
\end{tcolorbox}

    \begin{Verbatim}[commandchars=\\\{\}]
-1.7418198158396654
    \end{Verbatim}

    \hypertarget{fuxf3rmula-semiempuxedrica-de-la-masa}{%
\subsection{2. Fórmula semiempírica de la
masa}\label{fuxf3rmula-semiempuxedrica-de-la-masa}}

\textbf{La fórmula semiempírica de la masa (FSM)} En física nuclear, la
\textbf{fórmula de Weizsäcker} (conocida tambien como fórmula
semiempírica) sirve para evaluar la masa y otras propiedades de un
núcleo atómico; y está basada parcialmente en mediciones empíricas. En
particular la fórmula se usa para calcular la \textbf{\emph{energia de
enlace nuclear aproximada}} B, de un núcleo atómico con número atómico
\(Z\) y numero de masa \(A\):

\[
B = a_1 A - a_2 A^{2/3} - a_3 \frac{Z^2}{A^{1/3}}
- a_4 \frac{(A-2Z)^2}{A} + \frac{a_5}{A^{1/2}}
\]

donde, en unidades de millones de elctrón-volts, las constante son
\(a_1 = 15.8\), \(a_2 = 18.3\), \(a_3 = 0.714\), \(a_4 = 23.2\) y

\[
a_5 =
\begin{cases}
0 & \text{si A es impar},\\
12.0 & \text{si A y Z son pares(ambos),}\\
-12.0 & \text{si A es par y Z impar}
\end{cases}
\]

    \hypertarget{respuesta}{%
\subsection{RESPUESTA}\label{respuesta}}

    \emph{a)} Escribe un programa que tome como entrada los valores de \(A\)
y \(Z\), e imprima la energía de enlace \(B\) para el átomo
corespondiente. Usa tu programa para encontrar la energía de enlace de
un átomo con \(A = 58\) y \(Z = 28\) (Hint: La respuesta correcta es
alrededor de los 490 MeV)

    \begin{tcolorbox}[breakable, size=fbox, boxrule=1pt, pad at break*=1mm,colback=cellbackground, colframe=cellborder]
\prompt{In}{incolor}{26}{\boxspacing}
\begin{Verbatim}[commandchars=\\\{\}]
\PY{k}{def} \PY{n+nf}{energia\PYZus{}enlace}\PY{p}{(}\PY{n}{A}\PY{p}{,} \PY{n}{Z}\PY{p}{)}\PY{p}{:}
    \PY{n}{a1}\PY{p}{,} \PY{n}{a2}\PY{p}{,} \PY{n}{a3}\PY{p}{,} \PY{n}{a4} \PY{o}{=} \PY{l+m+mf}{15.8}\PY{p}{,} \PY{l+m+mf}{18.3}\PY{p}{,} \PY{l+m+mf}{0.714}\PY{p}{,} \PY{l+m+mf}{23.2}
    \PY{k}{if} \PY{n}{A} \PY{o}{\PYZpc{}} \PY{l+m+mi}{2} \PY{o}{==} \PY{l+m+mi}{1}\PY{p}{:}
        \PY{n}{a5} \PY{o}{=} \PY{l+m+mi}{0}
    \PY{k}{elif} \PY{n}{Z} \PY{o}{\PYZpc{}} \PY{l+m+mi}{2} \PY{o}{==} \PY{l+m+mi}{0}\PY{p}{:}
        \PY{n}{a5} \PY{o}{=} \PY{l+m+mf}{12.0}
    \PY{k}{else}\PY{p}{:}
        \PY{n}{a5} \PY{o}{=} \PY{o}{\PYZhy{}}\PY{l+m+mf}{12.0}
    \PY{n}{B} \PY{o}{=} \PY{p}{(}\PY{n}{a1}\PY{o}{*}\PY{n}{A} 
         \PY{o}{\PYZhy{}} \PY{n}{a2}\PY{o}{*}\PY{p}{(}\PY{n}{A}\PY{o}{*}\PY{o}{*}\PY{p}{(}\PY{l+m+mi}{2}\PY{o}{/}\PY{l+m+mi}{3}\PY{p}{)}\PY{p}{)} 
         \PY{o}{\PYZhy{}} \PY{n}{a3}\PY{o}{*}\PY{p}{(}\PY{n}{Z}\PY{o}{*}\PY{o}{*}\PY{l+m+mi}{2}\PY{p}{)}\PY{o}{/}\PY{p}{(}\PY{n}{A}\PY{o}{*}\PY{o}{*}\PY{p}{(}\PY{l+m+mi}{1}\PY{o}{/}\PY{l+m+mi}{3}\PY{p}{)}\PY{p}{)} 
         \PY{o}{\PYZhy{}} \PY{n}{a4}\PY{o}{*}\PY{p}{(}\PY{p}{(}\PY{n}{A} \PY{o}{\PYZhy{}} \PY{l+m+mi}{2}\PY{o}{*}\PY{n}{Z}\PY{p}{)}\PY{o}{*}\PY{o}{*}\PY{l+m+mi}{2}\PY{p}{)}\PY{o}{/}\PY{n}{A} 
         \PY{o}{+} \PY{n}{a5}\PY{o}{/}\PY{p}{(}\PY{n}{A}\PY{o}{*}\PY{o}{*}\PY{l+m+mf}{0.5}\PY{p}{)}\PY{p}{)}
    \PY{k}{return} \PY{n}{B}

\PY{k}{def} \PY{n+nf}{energia\PYZus{}por\PYZus{}nucleon}\PY{p}{(}\PY{n}{A}\PY{p}{,} \PY{n}{Z}\PY{p}{)}\PY{p}{:}
    \PY{k}{return} \PY{n}{energia\PYZus{}enlace}\PY{p}{(}\PY{n}{A}\PY{p}{,} \PY{n}{Z}\PY{p}{)} \PY{o}{/} \PY{n}{A}

\PY{k}{def} \PY{n+nf}{nucleo\PYZus{}mas\PYZus{}estable}\PY{p}{(}\PY{n}{Z}\PY{p}{)}\PY{p}{:}
    \PY{n}{mejor\PYZus{}A}\PY{p}{,} \PY{n}{mejor\PYZus{}BA} \PY{o}{=} \PY{n}{Z}\PY{p}{,} \PY{l+m+mi}{0}
    \PY{k}{for} \PY{n}{A} \PY{o+ow}{in} \PY{n+nb}{range}\PY{p}{(}\PY{n}{Z}\PY{p}{,} \PY{l+m+mi}{3}\PY{o}{*}\PY{n}{Z}\PY{o}{+}\PY{l+m+mi}{1}\PY{p}{)}\PY{p}{:}
        \PY{n}{BA} \PY{o}{=} \PY{n}{energia\PYZus{}por\PYZus{}nucleon}\PY{p}{(}\PY{n}{A}\PY{p}{,} \PY{n}{Z}\PY{p}{)}
        \PY{k}{if} \PY{n}{BA} \PY{o}{\PYZgt{}} \PY{n}{mejor\PYZus{}BA}\PY{p}{:}
            \PY{n}{mejor\PYZus{}A}\PY{p}{,} \PY{n}{mejor\PYZus{}BA} \PY{o}{=} \PY{n}{A}\PY{p}{,} \PY{n}{BA}
    \PY{k}{return} \PY{n}{mejor\PYZus{}A}\PY{p}{,} \PY{n}{mejor\PYZus{}BA}

\PY{n+nb}{print}\PY{p}{(}\PY{l+s+s2}{\PYZdq{}}\PY{l+s+s2}{a) Energía de enlace total (A=58, Z=28):}\PY{l+s+s2}{\PYZdq{}}\PY{p}{,} \PY{n}{energia\PYZus{}enlace}\PY{p}{(}\PY{l+m+mi}{58}\PY{p}{,}\PY{l+m+mi}{28}\PY{p}{)}\PY{p}{)}
\end{Verbatim}
\end{tcolorbox}

    \begin{Verbatim}[commandchars=\\\{\}]
a) Energía de enlace total (A=58, Z=28): 497.5620206224374
    \end{Verbatim}

    \emph{b)} Modifica el programa del inciso anterios, para escribir una
segunda versión que imprima no la energía de enlace total \(B\), sino la
energía de unión por mucleón que es \(B/A\)

    \begin{tcolorbox}[breakable, size=fbox, boxrule=1pt, pad at break*=1mm,colback=cellbackground, colframe=cellborder]
\prompt{In}{incolor}{27}{\boxspacing}
\begin{Verbatim}[commandchars=\\\{\}]
\PY{k}{def} \PY{n+nf}{energia\PYZus{}enlace}\PY{p}{(}\PY{n}{A}\PY{p}{,} \PY{n}{Z}\PY{p}{)}\PY{p}{:}
    \PY{n}{a1}\PY{p}{,} \PY{n}{a2}\PY{p}{,} \PY{n}{a3}\PY{p}{,} \PY{n}{a4} \PY{o}{=} \PY{l+m+mf}{15.8}\PY{p}{,} \PY{l+m+mf}{18.3}\PY{p}{,} \PY{l+m+mf}{0.714}\PY{p}{,} \PY{l+m+mf}{23.2}
    \PY{k}{if} \PY{n}{A} \PY{o}{\PYZpc{}} \PY{l+m+mi}{2} \PY{o}{==} \PY{l+m+mi}{1}\PY{p}{:}
        \PY{n}{a5} \PY{o}{=} \PY{l+m+mi}{0}
    \PY{k}{elif} \PY{n}{Z} \PY{o}{\PYZpc{}} \PY{l+m+mi}{2} \PY{o}{==} \PY{l+m+mi}{0}\PY{p}{:}
        \PY{n}{a5} \PY{o}{=} \PY{l+m+mf}{12.0}
    \PY{k}{else}\PY{p}{:}
        \PY{n}{a5} \PY{o}{=} \PY{o}{\PYZhy{}}\PY{l+m+mf}{12.0}
    \PY{n}{B} \PY{o}{=} \PY{p}{(}\PY{n}{a1}\PY{o}{*}\PY{n}{A} 
         \PY{o}{\PYZhy{}} \PY{n}{a2}\PY{o}{*}\PY{p}{(}\PY{n}{A}\PY{o}{*}\PY{o}{*}\PY{p}{(}\PY{l+m+mi}{2}\PY{o}{/}\PY{l+m+mi}{3}\PY{p}{)}\PY{p}{)} 
         \PY{o}{\PYZhy{}} \PY{n}{a3}\PY{o}{*}\PY{p}{(}\PY{n}{Z}\PY{o}{*}\PY{o}{*}\PY{l+m+mi}{2}\PY{p}{)}\PY{o}{/}\PY{p}{(}\PY{n}{A}\PY{o}{*}\PY{o}{*}\PY{p}{(}\PY{l+m+mi}{1}\PY{o}{/}\PY{l+m+mi}{3}\PY{p}{)}\PY{p}{)} 
         \PY{o}{\PYZhy{}} \PY{n}{a4}\PY{o}{*}\PY{p}{(}\PY{p}{(}\PY{n}{A} \PY{o}{\PYZhy{}} \PY{l+m+mi}{2}\PY{o}{*}\PY{n}{Z}\PY{p}{)}\PY{o}{*}\PY{o}{*}\PY{l+m+mi}{2}\PY{p}{)}\PY{o}{/}\PY{n}{A} 
         \PY{o}{+} \PY{n}{a5}\PY{o}{/}\PY{p}{(}\PY{n}{A}\PY{o}{*}\PY{o}{*}\PY{l+m+mf}{0.5}\PY{p}{)}\PY{p}{)}
    \PY{k}{return} \PY{n}{B}

\PY{k}{def} \PY{n+nf}{energia\PYZus{}por\PYZus{}nucleon}\PY{p}{(}\PY{n}{A}\PY{p}{,} \PY{n}{Z}\PY{p}{)}\PY{p}{:}
    \PY{k}{return} \PY{n}{energia\PYZus{}enlace}\PY{p}{(}\PY{n}{A}\PY{p}{,} \PY{n}{Z}\PY{p}{)} \PY{o}{/} \PY{n}{A}

\PY{k}{def} \PY{n+nf}{nucleo\PYZus{}mas\PYZus{}estable}\PY{p}{(}\PY{n}{Z}\PY{p}{)}\PY{p}{:}
    \PY{n}{mejor\PYZus{}A}\PY{p}{,} \PY{n}{mejor\PYZus{}BA} \PY{o}{=} \PY{n}{Z}\PY{p}{,} \PY{l+m+mi}{0}
    \PY{k}{for} \PY{n}{A} \PY{o+ow}{in} \PY{n+nb}{range}\PY{p}{(}\PY{n}{Z}\PY{p}{,} \PY{l+m+mi}{3}\PY{o}{*}\PY{n}{Z}\PY{o}{+}\PY{l+m+mi}{1}\PY{p}{)}\PY{p}{:}
        \PY{n}{BA} \PY{o}{=} \PY{n}{energia\PYZus{}por\PYZus{}nucleon}\PY{p}{(}\PY{n}{A}\PY{p}{,} \PY{n}{Z}\PY{p}{)}
        \PY{k}{if} \PY{n}{BA} \PY{o}{\PYZgt{}} \PY{n}{mejor\PYZus{}BA}\PY{p}{:}
            \PY{n}{mejor\PYZus{}A}\PY{p}{,} \PY{n}{mejor\PYZus{}BA} \PY{o}{=} \PY{n}{A}\PY{p}{,} \PY{n}{BA}
    \PY{k}{return} \PY{n}{mejor\PYZus{}A}\PY{p}{,} \PY{n}{mejor\PYZus{}BA}

\PY{n+nb}{print}\PY{p}{(}\PY{l+s+s2}{\PYZdq{}}\PY{l+s+s2}{b) Energía por nucleón (A=58, Z=28):}\PY{l+s+s2}{\PYZdq{}}\PY{p}{,} \PY{n}{energia\PYZus{}por\PYZus{}nucleon}\PY{p}{(}\PY{l+m+mi}{58}\PY{p}{,}\PY{l+m+mi}{28}\PY{p}{)}\PY{p}{)}
\end{Verbatim}
\end{tcolorbox}

    \begin{Verbatim}[commandchars=\\\{\}]
b) Energía por nucleón (A=58, Z=28): 8.578655527973059
    \end{Verbatim}

    \emph{c)} Escribe una tercera versión del programa para que tome como
entrada solo un valor del número atómico \(Z\) y luego pase por todos
los valores de \(A\) desde \(A = Z\) hasta \(A = 3Z\), para enconrar el
que iene la mayor energía de enlace por nucleón. Ese es el núcleo mas
estable con el número aómico dado. Haz que tu programa imprima el valor
de \(A\) para este núcleo más esable y el valor de la energía de enlace
por nucleón.

    \begin{tcolorbox}[breakable, size=fbox, boxrule=1pt, pad at break*=1mm,colback=cellbackground, colframe=cellborder]
\prompt{In}{incolor}{28}{\boxspacing}
\begin{Verbatim}[commandchars=\\\{\}]
\PY{k}{def} \PY{n+nf}{energia\PYZus{}enlace}\PY{p}{(}\PY{n}{A}\PY{p}{,} \PY{n}{Z}\PY{p}{)}\PY{p}{:}
    \PY{n}{a1}\PY{p}{,} \PY{n}{a2}\PY{p}{,} \PY{n}{a3}\PY{p}{,} \PY{n}{a4} \PY{o}{=} \PY{l+m+mf}{15.8}\PY{p}{,} \PY{l+m+mf}{18.3}\PY{p}{,} \PY{l+m+mf}{0.714}\PY{p}{,} \PY{l+m+mf}{23.2}
    \PY{k}{if} \PY{n}{A} \PY{o}{\PYZpc{}} \PY{l+m+mi}{2} \PY{o}{==} \PY{l+m+mi}{1}\PY{p}{:}
        \PY{n}{a5} \PY{o}{=} \PY{l+m+mi}{0}
    \PY{k}{elif} \PY{n}{Z} \PY{o}{\PYZpc{}} \PY{l+m+mi}{2} \PY{o}{==} \PY{l+m+mi}{0}\PY{p}{:}
        \PY{n}{a5} \PY{o}{=} \PY{l+m+mf}{12.0}
    \PY{k}{else}\PY{p}{:}
        \PY{n}{a5} \PY{o}{=} \PY{o}{\PYZhy{}}\PY{l+m+mf}{12.0}
    \PY{n}{B} \PY{o}{=} \PY{p}{(}\PY{n}{a1}\PY{o}{*}\PY{n}{A} 
         \PY{o}{\PYZhy{}} \PY{n}{a2}\PY{o}{*}\PY{p}{(}\PY{n}{A}\PY{o}{*}\PY{o}{*}\PY{p}{(}\PY{l+m+mi}{2}\PY{o}{/}\PY{l+m+mi}{3}\PY{p}{)}\PY{p}{)} 
         \PY{o}{\PYZhy{}} \PY{n}{a3}\PY{o}{*}\PY{p}{(}\PY{n}{Z}\PY{o}{*}\PY{o}{*}\PY{l+m+mi}{2}\PY{p}{)}\PY{o}{/}\PY{p}{(}\PY{n}{A}\PY{o}{*}\PY{o}{*}\PY{p}{(}\PY{l+m+mi}{1}\PY{o}{/}\PY{l+m+mi}{3}\PY{p}{)}\PY{p}{)} 
         \PY{o}{\PYZhy{}} \PY{n}{a4}\PY{o}{*}\PY{p}{(}\PY{p}{(}\PY{n}{A} \PY{o}{\PYZhy{}} \PY{l+m+mi}{2}\PY{o}{*}\PY{n}{Z}\PY{p}{)}\PY{o}{*}\PY{o}{*}\PY{l+m+mi}{2}\PY{p}{)}\PY{o}{/}\PY{n}{A} 
         \PY{o}{+} \PY{n}{a5}\PY{o}{/}\PY{p}{(}\PY{n}{A}\PY{o}{*}\PY{o}{*}\PY{l+m+mf}{0.5}\PY{p}{)}\PY{p}{)}
    \PY{k}{return} \PY{n}{B}

\PY{k}{def} \PY{n+nf}{energia\PYZus{}por\PYZus{}nucleon}\PY{p}{(}\PY{n}{A}\PY{p}{,} \PY{n}{Z}\PY{p}{)}\PY{p}{:}
    \PY{k}{return} \PY{n}{energia\PYZus{}enlace}\PY{p}{(}\PY{n}{A}\PY{p}{,} \PY{n}{Z}\PY{p}{)} \PY{o}{/} \PY{n}{A}

\PY{k}{def} \PY{n+nf}{nucleo\PYZus{}mas\PYZus{}estable}\PY{p}{(}\PY{n}{Z}\PY{p}{)}\PY{p}{:}
    \PY{n}{mejor\PYZus{}A}\PY{p}{,} \PY{n}{mejor\PYZus{}BA} \PY{o}{=} \PY{n}{Z}\PY{p}{,} \PY{l+m+mi}{0}
    \PY{k}{for} \PY{n}{A} \PY{o+ow}{in} \PY{n+nb}{range}\PY{p}{(}\PY{n}{Z}\PY{p}{,} \PY{l+m+mi}{3}\PY{o}{*}\PY{n}{Z}\PY{o}{+}\PY{l+m+mi}{1}\PY{p}{)}\PY{p}{:}
        \PY{n}{BA} \PY{o}{=} \PY{n}{energia\PYZus{}por\PYZus{}nucleon}\PY{p}{(}\PY{n}{A}\PY{p}{,} \PY{n}{Z}\PY{p}{)}
        \PY{k}{if} \PY{n}{BA} \PY{o}{\PYZgt{}} \PY{n}{mejor\PYZus{}BA}\PY{p}{:}
            \PY{n}{mejor\PYZus{}A}\PY{p}{,} \PY{n}{mejor\PYZus{}BA} \PY{o}{=} \PY{n}{A}\PY{p}{,} \PY{n}{BA}
    \PY{k}{return} \PY{n}{mejor\PYZus{}A}\PY{p}{,} \PY{n}{mejor\PYZus{}BA}

\PY{n+nb}{print}\PY{p}{(}\PY{l+s+s2}{\PYZdq{}}\PY{l+s+s2}{c) Núcleo más estable para Z=28:}\PY{l+s+s2}{\PYZdq{}}\PY{p}{,} \PY{n}{nucleo\PYZus{}mas\PYZus{}estable}\PY{p}{(}\PY{l+m+mi}{28}\PY{p}{)}\PY{p}{)}
\end{Verbatim}
\end{tcolorbox}

    \begin{Verbatim}[commandchars=\\\{\}]
c) Núcleo más estable para Z=28: (62, 8.70245768367189)
    \end{Verbatim}

    \emph{d)} FInalmente, escribe una cuarta versión del programa que, en
lugar de tomar \(Z\) como entrada, se ejecute a través de todos los
valores de \(Z\) de 1 a 100 e imprima el valor ,ás estable de \(A\) para
cada uno. ¿A qué valor de \(Z\) se produce la energía de enlace máxima
por nucleón? (La respuesta correcta, en la vida real, es \(Z = 28\), que
corresponde al Níquel).

    \begin{tcolorbox}[breakable, size=fbox, boxrule=1pt, pad at break*=1mm,colback=cellbackground, colframe=cellborder]
\prompt{In}{incolor}{29}{\boxspacing}
\begin{Verbatim}[commandchars=\\\{\}]
\PY{k}{def} \PY{n+nf}{energia\PYZus{}enlace}\PY{p}{(}\PY{n}{A}\PY{p}{,} \PY{n}{Z}\PY{p}{)}\PY{p}{:}
    \PY{n}{a1}\PY{p}{,} \PY{n}{a2}\PY{p}{,} \PY{n}{a3}\PY{p}{,} \PY{n}{a4} \PY{o}{=} \PY{l+m+mf}{15.8}\PY{p}{,} \PY{l+m+mf}{18.3}\PY{p}{,} \PY{l+m+mf}{0.714}\PY{p}{,} \PY{l+m+mf}{23.2}
    \PY{k}{if} \PY{n}{A} \PY{o}{\PYZpc{}} \PY{l+m+mi}{2} \PY{o}{==} \PY{l+m+mi}{1}\PY{p}{:}
        \PY{n}{a5} \PY{o}{=} \PY{l+m+mi}{0}
    \PY{k}{elif} \PY{n}{Z} \PY{o}{\PYZpc{}} \PY{l+m+mi}{2} \PY{o}{==} \PY{l+m+mi}{0}\PY{p}{:}
        \PY{n}{a5} \PY{o}{=} \PY{l+m+mf}{12.0}
    \PY{k}{else}\PY{p}{:}
        \PY{n}{a5} \PY{o}{=} \PY{o}{\PYZhy{}}\PY{l+m+mf}{12.0}
    \PY{n}{B} \PY{o}{=} \PY{p}{(}\PY{n}{a1}\PY{o}{*}\PY{n}{A} 
         \PY{o}{\PYZhy{}} \PY{n}{a2}\PY{o}{*}\PY{p}{(}\PY{n}{A}\PY{o}{*}\PY{o}{*}\PY{p}{(}\PY{l+m+mi}{2}\PY{o}{/}\PY{l+m+mi}{3}\PY{p}{)}\PY{p}{)} 
         \PY{o}{\PYZhy{}} \PY{n}{a3}\PY{o}{*}\PY{p}{(}\PY{n}{Z}\PY{o}{*}\PY{o}{*}\PY{l+m+mi}{2}\PY{p}{)}\PY{o}{/}\PY{p}{(}\PY{n}{A}\PY{o}{*}\PY{o}{*}\PY{p}{(}\PY{l+m+mi}{1}\PY{o}{/}\PY{l+m+mi}{3}\PY{p}{)}\PY{p}{)} 
         \PY{o}{\PYZhy{}} \PY{n}{a4}\PY{o}{*}\PY{p}{(}\PY{p}{(}\PY{n}{A} \PY{o}{\PYZhy{}} \PY{l+m+mi}{2}\PY{o}{*}\PY{n}{Z}\PY{p}{)}\PY{o}{*}\PY{o}{*}\PY{l+m+mi}{2}\PY{p}{)}\PY{o}{/}\PY{n}{A} 
         \PY{o}{+} \PY{n}{a5}\PY{o}{/}\PY{p}{(}\PY{n}{A}\PY{o}{*}\PY{o}{*}\PY{l+m+mf}{0.5}\PY{p}{)}\PY{p}{)}
    \PY{k}{return} \PY{n}{B}

\PY{k}{def} \PY{n+nf}{energia\PYZus{}por\PYZus{}nucleon}\PY{p}{(}\PY{n}{A}\PY{p}{,} \PY{n}{Z}\PY{p}{)}\PY{p}{:}
    \PY{k}{return} \PY{n}{energia\PYZus{}enlace}\PY{p}{(}\PY{n}{A}\PY{p}{,} \PY{n}{Z}\PY{p}{)} \PY{o}{/} \PY{n}{A}

\PY{k}{def} \PY{n+nf}{nucleo\PYZus{}mas\PYZus{}estable}\PY{p}{(}\PY{n}{Z}\PY{p}{)}\PY{p}{:}
    \PY{n}{mejor\PYZus{}A}\PY{p}{,} \PY{n}{mejor\PYZus{}BA} \PY{o}{=} \PY{n}{Z}\PY{p}{,} \PY{l+m+mi}{0}
    \PY{k}{for} \PY{n}{A} \PY{o+ow}{in} \PY{n+nb}{range}\PY{p}{(}\PY{n}{Z}\PY{p}{,} \PY{l+m+mi}{3}\PY{o}{*}\PY{n}{Z}\PY{o}{+}\PY{l+m+mi}{1}\PY{p}{)}\PY{p}{:}
        \PY{n}{BA} \PY{o}{=} \PY{n}{energia\PYZus{}por\PYZus{}nucleon}\PY{p}{(}\PY{n}{A}\PY{p}{,} \PY{n}{Z}\PY{p}{)}
        \PY{k}{if} \PY{n}{BA} \PY{o}{\PYZgt{}} \PY{n}{mejor\PYZus{}BA}\PY{p}{:}
            \PY{n}{mejor\PYZus{}A}\PY{p}{,} \PY{n}{mejor\PYZus{}BA} \PY{o}{=} \PY{n}{A}\PY{p}{,} \PY{n}{BA}
    \PY{k}{return} \PY{n}{mejor\PYZus{}A}\PY{p}{,} \PY{n}{mejor\PYZus{}BA}
\end{Verbatim}
\end{tcolorbox}

    \begin{tcolorbox}[breakable, size=fbox, boxrule=1pt, pad at break*=1mm,colback=cellbackground, colframe=cellborder]
\prompt{In}{incolor}{30}{\boxspacing}
\begin{Verbatim}[commandchars=\\\{\}]
\PY{n}{resultados} \PY{o}{=} \PY{p}{[}\PY{p}{]}
\PY{k}{for} \PY{n}{Z} \PY{o+ow}{in} \PY{n+nb}{range}\PY{p}{(}\PY{l+m+mi}{1}\PY{p}{,} \PY{l+m+mi}{101}\PY{p}{)}\PY{p}{:}
    \PY{n}{A\PYZus{}estable}\PY{p}{,} \PY{n}{BA} \PY{o}{=} \PY{n}{nucleo\PYZus{}mas\PYZus{}estable}\PY{p}{(}\PY{n}{Z}\PY{p}{)}
    \PY{n}{resultados}\PY{o}{.}\PY{n}{append}\PY{p}{(}\PY{p}{(}\PY{n}{Z}\PY{p}{,} \PY{n}{A\PYZus{}estable}\PY{p}{,} \PY{n}{BA}\PY{p}{)}\PY{p}{)}

\PY{n+nb}{print}\PY{p}{(}\PY{l+s+s2}{\PYZdq{}}\PY{l+s+se}{\PYZbs{}n}\PY{l+s+s2}{d) Resultados cada 10 elementos:}\PY{l+s+s2}{\PYZdq{}}\PY{p}{)}
\PY{k}{for} \PY{n}{Z}\PY{p}{,} \PY{n}{A}\PY{p}{,} \PY{n}{BA} \PY{o+ow}{in} \PY{n}{resultados}\PY{p}{[}\PY{p}{:}\PY{p}{:}\PY{l+m+mi}{10}\PY{p}{]}\PY{p}{:}
    \PY{n+nb}{print}\PY{p}{(}\PY{l+s+sa}{f}\PY{l+s+s2}{\PYZdq{}}\PY{l+s+s2}{Z=}\PY{l+s+si}{\PYZob{}}\PY{n}{Z}\PY{l+s+si}{\PYZcb{}}\PY{l+s+s2}{, A más estable=}\PY{l+s+si}{\PYZob{}}\PY{n}{A}\PY{l+s+si}{\PYZcb{}}\PY{l+s+s2}{, B/A=}\PY{l+s+si}{\PYZob{}}\PY{n}{BA}\PY{l+s+si}{:}\PY{l+s+s2}{.2f}\PY{l+s+si}{\PYZcb{}}\PY{l+s+s2}{\PYZdq{}}\PY{p}{)}
\end{Verbatim}
\end{tcolorbox}

    \begin{Verbatim}[commandchars=\\\{\}]

d) Resultados cada 10 elementos:
Z=1, A más estable=3, B/A=0.37
Z=11, A más estable=25, B/A=8.03
Z=21, A más estable=47, B/A=8.61
Z=31, A más estable=69, B/A=8.68
Z=41, A más estable=93, B/A=8.59
Z=51, A más estable=119, B/A=8.43
Z=61, A más estable=143, B/A=8.25
Z=71, A más estable=169, B/A=8.05
Z=81, A más estable=195, B/A=7.84
Z=91, A más estable=223, B/A=7.63
    \end{Verbatim}

    \hypertarget{coeficientes-binomiales}{%
\subsection{3. Coeficientes binomiales}\label{coeficientes-binomiales}}

El coeficiente binomial \(\binom{n}{k}\) es un número entero igual a:

\[
\binom{n}{k} = \frac{n!}{k!(n-k)!} = \frac{n \times (n - 1) \times (n - 2) \times \dots \times (n - k + 1)}{1 \times 2 \times \dots \times k}
\]

donde \(k \geq 1\), o bien \(\binom{n}{0} = 1\) cuando \(k = 0\)

    \hypertarget{respuesta}{%
\subsection{RESPUESTA}\label{respuesta}}

    \emph{a)} Utiliza esta formula para escribir una función llamada
\(\texttt{binomial(n, k)}\) (o como tu quieras) que calcule el
coeficiente binomial para un \(n\) y \(k\) dados. Asegúrate de que tu
función devuelva la respuesta en forma de un número entero (no flotante)
y proporcione el valor correcto de 1 para el caso en que \(k = 0\)

    \begin{tcolorbox}[breakable, size=fbox, boxrule=1pt, pad at break*=1mm,colback=cellbackground, colframe=cellborder]
\prompt{In}{incolor}{31}{\boxspacing}
\begin{Verbatim}[commandchars=\\\{\}]
\PY{k+kn}{import} \PY{n+nn}{math}

\PY{k}{def} \PY{n+nf}{binomial}\PY{p}{(}\PY{n}{n}\PY{p}{,} \PY{n}{k}\PY{p}{)}\PY{p}{:}
    \PY{k}{return} \PY{n}{math}\PY{o}{.}\PY{n}{comb}\PY{p}{(}\PY{n}{n}\PY{p}{,} \PY{n}{k}\PY{p}{)}
    \PY{k}{if} \PY{n}{k} \PY{o}{==} \PY{l+m+mi}{0} \PY{o+ow}{or} \PY{n}{k} \PY{o}{==} \PY{n}{n}\PY{p}{:}
        \PY{k}{return} \PY{l+m+mi}{1}
    \PY{k}{else}\PY{p}{:}
        \PY{k}{return} \PY{n}{math}\PY{o}{.}\PY{n}{factorial}\PY{p}{(}\PY{n}{n}\PY{p}{)} \PY{o}{/}\PY{o}{/} \PY{p}{(}\PY{n}{math}\PY{o}{.}\PY{n}{factorial}\PY{p}{(}\PY{n}{k}\PY{p}{)} \PY{o}{*} \PY{n}{math}\PY{o}{.}\PY{n}{factorial}\PY{p}{(}\PY{n}{n} \PY{o}{\PYZhy{}} \PY{n}{k}\PY{p}{)}\PY{p}{)}

\PY{n+nb}{print}\PY{p}{(}\PY{n}{binomial}\PY{p}{(}\PY{l+m+mi}{5}\PY{p}{,} \PY{l+m+mi}{2}\PY{p}{)}\PY{p}{)}
\end{Verbatim}
\end{tcolorbox}

    \begin{Verbatim}[commandchars=\\\{\}]
10
    \end{Verbatim}

    \emph{b)} Usando tu función, escribe un programa que imprima las
primeras 20 lineas del ``\emph{triángulo de Pascal}''. La \emph{n}-ésima
línea del triángulo de Pascal contiene \(n + 1\) números, que son los
coeficientes \(\binom{n}{0}\), \(\binom{n}{1}\), y así sucesivamente
hasta \(\binom{n}{n}\). De tal manera que las primeras líneas son:

    \begin{tcolorbox}[breakable, size=fbox, boxrule=1pt, pad at break*=1mm,colback=cellbackground, colframe=cellborder]
\prompt{In}{incolor}{32}{\boxspacing}
\begin{Verbatim}[commandchars=\\\{\}]
\PY{k+kn}{import} \PY{n+nn}{math}

\PY{k}{def} \PY{n+nf}{binomial}\PY{p}{(}\PY{n}{n}\PY{p}{,} \PY{n}{k}\PY{p}{)}\PY{p}{:}
    \PY{k}{return} \PY{n}{math}\PY{o}{.}\PY{n}{comb}\PY{p}{(}\PY{n}{n}\PY{p}{,} \PY{n}{k}\PY{p}{)}
\PY{n+nb}{print}\PY{p}{(}\PY{l+s+s2}{\PYZdq{}}\PY{l+s+s2}{b) Triángulo de Pascal (20 líneas):}\PY{l+s+s2}{\PYZdq{}}\PY{p}{)}
\PY{k}{for} \PY{n}{n} \PY{o+ow}{in} \PY{n+nb}{range}\PY{p}{(}\PY{l+m+mi}{20}\PY{p}{)}\PY{p}{:}
    \PY{n}{fila} \PY{o}{=} \PY{p}{[}\PY{n}{binomial}\PY{p}{(}\PY{n}{n}\PY{p}{,} \PY{n}{k}\PY{p}{)} \PY{k}{for} \PY{n}{k} \PY{o+ow}{in} \PY{n+nb}{range}\PY{p}{(}\PY{n}{n}\PY{o}{+}\PY{l+m+mi}{1}\PY{p}{)}\PY{p}{]}
    \PY{n+nb}{print}\PY{p}{(}\PY{l+s+s2}{\PYZdq{}}\PY{l+s+s2}{ }\PY{l+s+s2}{\PYZdq{}}\PY{o}{.}\PY{n}{join}\PY{p}{(}\PY{n+nb}{map}\PY{p}{(}\PY{n+nb}{str}\PY{p}{,} \PY{n}{fila}\PY{p}{)}\PY{p}{)}\PY{p}{)}
\end{Verbatim}
\end{tcolorbox}

    \begin{Verbatim}[commandchars=\\\{\}]
b) Triángulo de Pascal (20 líneas):
1
1 1
1 2 1
1 3 3 1
1 4 6 4 1
1 5 10 10 5 1
1 6 15 20 15 6 1
1 7 21 35 35 21 7 1
1 8 28 56 70 56 28 8 1
1 9 36 84 126 126 84 36 9 1
1 10 45 120 210 252 210 120 45 10 1
1 11 55 165 330 462 462 330 165 55 11 1
1 12 66 220 495 792 924 792 495 220 66 12 1
1 13 78 286 715 1287 1716 1716 1287 715 286 78 13 1
1 14 91 364 1001 2002 3003 3432 3003 2002 1001 364 91 14 1
1 15 105 455 1365 3003 5005 6435 6435 5005 3003 1365 455 105 15 1
1 16 120 560 1820 4368 8008 11440 12870 11440 8008 4368 1820 560 120 16 1
1 17 136 680 2380 6188 12376 19448 24310 24310 19448 12376 6188 2380 680 136 17
1
1 18 153 816 3060 8568 18564 31824 43758 48620 43758 31824 18564 8568 3060 816
153 18 1
1 19 171 969 3876 11628 27132 50388 75582 92378 92378 75582 50388 27132 11628
3876 969 171 19 1
    \end{Verbatim}

    \emph{c)} La probabilidad de que para una moneda no sesgada, lanzada
\(n\) veces, salga águila \(k\) veces es:

\[
p(k|n) = \frac{\binom{n}{k}}{2^n}
\]

Escribe un programa para calcular:

    \begin{enumerate}
\def\labelenumi{\arabic{enumi})}
\tightlist
\item
  La probabilidad total de que una moneda lanzada 100 veces, salga
  águila exactamente 60 veces
\end{enumerate}

    \begin{tcolorbox}[breakable, size=fbox, boxrule=1pt, pad at break*=1mm,colback=cellbackground, colframe=cellborder]
\prompt{In}{incolor}{33}{\boxspacing}
\begin{Verbatim}[commandchars=\\\{\}]
\PY{k}{def} \PY{n+nf}{probabilidad\PYZus{}1}\PY{p}{(}\PY{n}{n}\PY{p}{,} \PY{n}{k}\PY{p}{)}\PY{p}{:}
    \PY{k}{return} \PY{n}{binomial}\PY{p}{(}\PY{n}{n}\PY{p}{,} \PY{n}{k}\PY{p}{)} \PY{o}{/} \PY{p}{(}\PY{l+m+mi}{2}\PY{o}{*}\PY{o}{*}\PY{n}{n}\PY{p}{)}

\PY{n}{p1} \PY{o}{=} \PY{n}{probabilidad}\PY{p}{(}\PY{l+m+mi}{100}\PY{p}{,} \PY{l+m+mi}{60}\PY{p}{)}

\PY{n+nb}{print}\PY{p}{(}\PY{l+s+s2}{\PYZdq{}}\PY{l+s+s2}{P(60 águilas en 100):}\PY{l+s+s2}{\PYZdq{}}\PY{p}{,} \PY{n}{p1}\PY{p}{)}
\end{Verbatim}
\end{tcolorbox}

    \begin{Verbatim}[commandchars=\\\{\}, frame=single, framerule=2mm, rulecolor=\color{outerrorbackground}]
\textcolor{ansi-red}{---------------------------------------------------------------------------}
\textcolor{ansi-red}{NameError}                                 Traceback (most recent call last)
\textcolor{ansi-green}{/tmp/ipykernel\_16225/2667472364.py} in \textcolor{ansi-cyan}{<module>}
\textcolor{ansi-green-intense}{\textbf{      2}}     \textcolor{ansi-green}{return} binomial\textcolor{ansi-blue}{(}n\textcolor{ansi-blue}{,} k\textcolor{ansi-blue}{)} \textcolor{ansi-blue}{/} \textcolor{ansi-blue}{(}\textcolor{ansi-cyan}{2}\textcolor{ansi-blue}{**}n\textcolor{ansi-blue}{)}
\textcolor{ansi-green-intense}{\textbf{      3}} 
\textcolor{ansi-green}{----> 4}\textcolor{ansi-red}{ }p1 \textcolor{ansi-blue}{=} probabilidad\textcolor{ansi-blue}{(}\textcolor{ansi-cyan}{100}\textcolor{ansi-blue}{,} \textcolor{ansi-cyan}{60}\textcolor{ansi-blue}{)}
\textcolor{ansi-green-intense}{\textbf{      5}} 
\textcolor{ansi-green-intense}{\textbf{      6}} print\textcolor{ansi-blue}{(}\textcolor{ansi-blue}{"P(60 águilas en 100):"}\textcolor{ansi-blue}{,} p1\textcolor{ansi-blue}{)}

\textcolor{ansi-red}{NameError}: name 'probabilidad' is not defined
    \end{Verbatim}

    \begin{enumerate}
\def\labelenumi{\arabic{enumi})}
\setcounter{enumi}{1}
\tightlist
\item
  La probabilidad de que salga águila 60 veces o más
\end{enumerate}

    \begin{tcolorbox}[breakable, size=fbox, boxrule=1pt, pad at break*=1mm,colback=cellbackground, colframe=cellborder]
\prompt{In}{incolor}{ }{\boxspacing}
\begin{Verbatim}[commandchars=\\\{\}]
\PY{k}{def} \PY{n+nf}{probabilidad\PYZus{}2}\PY{p}{(}\PY{n}{n}\PY{p}{,} \PY{n}{k}\PY{p}{)}\PY{p}{:}
    \PY{n}{prob} \PY{o}{=} \PY{l+m+mi}{0}
    \PY{k}{for} \PY{n}{i} \PY{o+ow}{in} \PY{n+nb}{range}\PY{p}{(}\PY{n}{k}\PY{p}{,} \PY{n}{n} \PY{o}{+} \PY{l+m+mi}{1}\PY{p}{)}\PY{p}{:}
        \PY{n}{prob} \PY{o}{+}\PY{o}{=} \PY{n}{probabilidad\PYZus{}1}\PY{p}{(}\PY{n}{n}\PY{p}{,} \PY{n}{i}\PY{p}{)}
        \PY{k}{return} \PY{n}{prob}

\PY{n}{p2} \PY{o}{=} \PY{n+nb}{sum}\PY{p}{(}\PY{n}{probabilidad}\PY{p}{(}\PY{l+m+mi}{100}\PY{p}{,} \PY{n}{k}\PY{p}{)} \PY{k}{for} \PY{n}{k} \PY{o+ow}{in} \PY{n+nb}{range}\PY{p}{(}\PY{l+m+mi}{60}\PY{p}{,} \PY{l+m+mi}{101}\PY{p}{)}\PY{p}{)}

\PY{n+nb}{print}\PY{p}{(}\PY{l+s+s2}{\PYZdq{}}\PY{l+s+s2}{P(60 o +60):}\PY{l+s+s2}{\PYZdq{}}\PY{p}{,} \PY{n}{p2}\PY{p}{)}
\end{Verbatim}
\end{tcolorbox}

    \hypertarget{nuxfameros-primos}{%
\subsection{4. Números primos}\label{nuxfameros-primos}}

Una manera no muy eficiente para calcular números primos, es comprobar i
cada número es divisible por cualquier número menor que él. Sin embargo,
es posible escribir un programa mucho más rápido para números primos
utilizando las siguientes observaciones:

\emph{a)} Un número \(n\) es primo si no tiene factores primos menores
que \(n\). Por lo tanto, solo necesitamos comprobar si es divisible por
otros primos.

\emph{b)} Si un número no es primo, con un factor \(r\), entonces
\(n = rs\), donde \(s\) tambien es un factor. Si \(r \geq \sqrt{n}\),
entonce \(n = rs \ geq \sqrt{ns}\), lo que implica que
\(s \leq \sqrt{n}\). En otras palabras, cualquier número no primo debe
tener factores, y por lo tanto tambien factores primos, menores o
iguales a \(\sqrt{n}\). Por lo tanto, para determinar si un número es
primo, debemos comprobar sus factores primos solo hasta \(\sqrt{n}\)
inclusiv; si no hay ninguno, entonces el número es primo.

\emph{c)} Si encontramos incluso un solo factor primo menor que
\(\sqrt{n}\), sabemos que el número no es primo y, por lo tanto, no hay
necesidad de comprobar más; podemos descartar este número y pasar a
otro.

    Escribe un programa que encuentre todos los primos hasta diez mil. Crea
una lista para almacenar los primos, que comience sólo con el número 2.
Luego para cada número \(n\) del 3 al 10,000 comprueba si es divisible
por alguno de los primos de la lista hasta \(\sqrt{n}\). En cuanto
encuentres un facotr primo, puedes dejar de revisar los demas; pues ya
sabes que \(n\) no es primo.

SI no encuentras ningún factor primo \(\sqrt{n}\) o menor, entonces
\(n\) es primo y debes añadirlo a la lista.

Puedes imprimir la lista completa al final del programa o imprimir los
numeros individuales a medida que los encuentras.

    \hypertarget{respuesta}{%
\subsection{RESPUESTA}\label{respuesta}}

    \begin{tcolorbox}[breakable, size=fbox, boxrule=1pt, pad at break*=1mm,colback=cellbackground, colframe=cellborder]
\prompt{In}{incolor}{ }{\boxspacing}
\begin{Verbatim}[commandchars=\\\{\}]
\PY{k+kn}{import} \PY{n+nn}{math}

\PY{c+c1}{\PYZsh{}Encontrar todos los primos hasta diez mil}
\PY{k}{def} \PY{n+nf}{main}\PY{p}{(}\PY{p}{)}\PY{p}{:}
    \PY{n}{primos} \PY{o}{=} \PY{n}{encontrar\PYZus{}primos\PYZus{}hasta}\PY{p}{(}\PY{l+m+mi}{10000}\PY{p}{)}
    \PY{n+nb}{print}\PY{p}{(}\PY{l+s+sa}{f}\PY{l+s+s2}{\PYZdq{}}\PY{l+s+s2}{Se encontraron }\PY{l+s+si}{\PYZob{}}\PY{n+nb}{len}\PY{p}{(}\PY{n}{primos}\PY{p}{)}\PY{l+s+si}{\PYZcb{}}\PY{l+s+s2}{ números primos hasta 10,000}\PY{l+s+s2}{\PYZdq{}}\PY{p}{)}
    \PY{n+nb}{print}\PY{p}{(}\PY{n}{primos}\PY{p}{)}

\PY{k}{if} \PY{n+nv+vm}{\PYZus{}\PYZus{}name\PYZus{}\PYZus{}} \PY{o}{==} \PY{l+s+s2}{\PYZdq{}}\PY{l+s+s2}{\PYZus{}\PYZus{}main\PYZus{}\PYZus{}}\PY{l+s+s2}{\PYZdq{}}\PY{p}{:}
    \PY{n}{main}\PY{p}{(}\PY{p}{)}
\end{Verbatim}
\end{tcolorbox}

    \begin{tcolorbox}[breakable, size=fbox, boxrule=1pt, pad at break*=1mm,colback=cellbackground, colframe=cellborder]
\prompt{In}{incolor}{ }{\boxspacing}
\begin{Verbatim}[commandchars=\\\{\}]
\PY{k+kn}{import} \PY{n+nn}{math}

\PY{c+c1}{\PYZsh{} Creamos una lista con el número primo 2}
\PY{n}{primos} \PY{o}{=} \PY{p}{[}\PY{l+m+mi}{2}\PY{p}{]}

\PY{k}{def} \PY{n+nf}{es\PYZus{}primo}\PY{p}{(}\PY{n}{n}\PY{p}{)}\PY{p}{:}
    \PY{c+c1}{\PYZsh{} Verificamos divisibilidad por los primos en la lista hasta la raíz cuadrada de n}
    \PY{n}{raiz} \PY{o}{=} \PY{n}{math}\PY{o}{.}\PY{n}{isqrt}\PY{p}{(}\PY{n}{n}\PY{p}{)}
    \PY{k}{for} \PY{n}{primo} \PY{o+ow}{in} \PY{n}{primos}\PY{p}{:}
        \PY{k}{if} \PY{n}{primo} \PY{o}{\PYZgt{}} \PY{n}{raiz}\PY{p}{:}
            \PY{k}{break}
        \PY{k}{if} \PY{n}{n} \PY{o}{\PYZpc{}} \PY{n}{primo} \PY{o}{==} \PY{l+m+mi}{0}\PY{p}{:}
            \PY{k}{return} \PY{k+kc}{False}
    \PY{k}{return} \PY{k+kc}{True}

\PY{c+c1}{\PYZsh{} Encontrar todos los primos hasta 10000}
\PY{k}{for} \PY{n}{numero} \PY{o+ow}{in} \PY{n+nb}{range}\PY{p}{(}\PY{l+m+mi}{3}\PY{p}{,} \PY{l+m+mi}{10001}\PY{p}{)}\PY{p}{:}
    \PY{k}{if} \PY{n}{es\PYZus{}primo}\PY{p}{(}\PY{n}{numero}\PY{p}{)}\PY{p}{:}
        \PY{n}{primos}\PY{o}{.}\PY{n}{append}\PY{p}{(}\PY{n}{numero}\PY{p}{)}

\PY{c+c1}{\PYZsh{} Filtrar solo los primos que empiezan con 2}
\PY{n}{primos\PYZus{}con\PYZus{}2} \PY{o}{=} \PY{p}{[}\PY{p}{]}
\PY{k}{for} \PY{n}{primo} \PY{o+ow}{in} \PY{n}{primos}\PY{p}{:}
    \PY{k}{if} \PY{n+nb}{str}\PY{p}{(}\PY{n}{primo}\PY{p}{)}\PY{o}{.}\PY{n}{startswith}\PY{p}{(}\PY{l+s+s1}{\PYZsq{}}\PY{l+s+s1}{2}\PY{l+s+s1}{\PYZsq{}}\PY{p}{)}\PY{p}{:}
        \PY{n}{primos\PYZus{}con\PYZus{}2}\PY{o}{.}\PY{n}{append}\PY{p}{(}\PY{n}{primo}\PY{p}{)}

\PY{c+c1}{\PYZsh{} Guardar en archivo .txt}
\PY{k}{with} \PY{n+nb}{open}\PY{p}{(}\PY{l+s+s1}{\PYZsq{}}\PY{l+s+s1}{primos\PYZus{}con\PYZus{}2.txt}\PY{l+s+s1}{\PYZsq{}}\PY{p}{,} \PY{l+s+s1}{\PYZsq{}}\PY{l+s+s1}{w}\PY{l+s+s1}{\PYZsq{}}\PY{p}{)} \PY{k}{as} \PY{n}{archivo}\PY{p}{:}
    \PY{k}{for} \PY{n}{primo} \PY{o+ow}{in} \PY{n}{primos\PYZus{}con\PYZus{}2}\PY{p}{:}
        \PY{n}{archivo}\PY{o}{.}\PY{n}{write}\PY{p}{(}\PY{n+nb}{str}\PY{p}{(}\PY{n}{primo}\PY{p}{)} \PY{o}{+} \PY{l+s+s1}{\PYZsq{}}\PY{l+s+se}{\PYZbs{}n}\PY{l+s+s1}{\PYZsq{}}\PY{p}{)}

\PY{n+nb}{print}\PY{p}{(}\PY{l+s+s2}{\PYZdq{}}\PY{l+s+s2}{¡Archivo creado exitosamente!}\PY{l+s+s2}{\PYZdq{}}\PY{p}{)}
\end{Verbatim}
\end{tcolorbox}

    \begin{tcolorbox}[breakable, size=fbox, boxrule=1pt, pad at break*=1mm,colback=cellbackground, colframe=cellborder]
\prompt{In}{incolor}{ }{\boxspacing}
\begin{Verbatim}[commandchars=\\\{\}]
\PY{k}{for} \PY{n}{n} \PY{o+ow}{in} \PY{n+nb}{range}\PY{p}{(}\PY{l+m+mi}{3}\PY{p}{,} \PY{l+m+mi}{10001}\PY{p}{,} \PY{l+m+mi}{2}\PY{p}{)}\PY{p}{:}  \PY{c+c1}{\PYZsh{} Comenzamos desde 3 y saltamos de 2 en 2}
    \PY{k}{if} \PY{n}{es\PYZus{}primo}\PY{p}{(}\PY{n}{n}\PY{p}{)}\PY{p}{:}
        \PY{n}{primos}\PY{o}{.}\PY{n}{append}\PY{p}{(}\PY{n}{n}\PY{p}{)}

\PY{k}{for} \PY{n}{primo} \PY{o+ow}{in} \PY{n}{primos}\PY{p}{:}
    \PY{n+nb}{print}\PY{p}{(}\PY{n}{primo}\PY{p}{)}
\end{Verbatim}
\end{tcolorbox}

    A

    \begin{tcolorbox}[breakable, size=fbox, boxrule=1pt, pad at break*=1mm,colback=cellbackground, colframe=cellborder]
\prompt{In}{incolor}{ }{\boxspacing}
\begin{Verbatim}[commandchars=\\\{\}]
\PY{k+kn}{import} \PY{n+nn}{math}

\PY{k}{def} \PY{n+nf}{encontrar\PYZus{}primos\PYZus{}hasta}\PY{p}{(}\PY{n}{limite}\PY{p}{)}\PY{p}{:}
    \PY{l+s+sd}{\PYZdq{}\PYZdq{}\PYZdq{}}
\PY{l+s+sd}{    Encuentra todos los números primos hasta el límite especificado}
\PY{l+s+sd}{    \PYZdq{}\PYZdq{}\PYZdq{}}
    \PY{c+c1}{\PYZsh{} Creamos una lista que comienza sólo con el número primo 2}
    \PY{n}{primos} \PY{o}{=} \PY{p}{[}\PY{l+m+mi}{2}\PY{p}{]}
    
    \PY{c+c1}{\PYZsh{} Para cada número n del 3 al límite}
    \PY{k}{for} \PY{n}{n} \PY{o+ow}{in} \PY{n+nb}{range}\PY{p}{(}\PY{l+m+mi}{3}\PY{p}{,} \PY{n}{limite} \PY{o}{+} \PY{l+m+mi}{1}\PY{p}{)}\PY{p}{:}
        \PY{c+c1}{\PYZsh{} Calculamos la raíz cuadrada de n}
        \PY{n}{raiz\PYZus{}n} \PY{o}{=} \PY{n}{math}\PY{o}{.}\PY{n}{isqrt}\PY{p}{(}\PY{n}{n}\PY{p}{)}
        \PY{n}{es\PYZus{}primo} \PY{o}{=} \PY{k+kc}{True}
        
        \PY{c+c1}{\PYZsh{} Comprobamos si es divisible por alguno de los primos de la lista hasta √n}
        \PY{k}{for} \PY{n}{primo} \PY{o+ow}{in} \PY{n}{primos}\PY{p}{:}
            \PY{c+c1}{\PYZsh{} Si el primo actual es mayor que √n, podemos detenernos}
            \PY{k}{if} \PY{n}{primo} \PY{o}{\PYZgt{}} \PY{n}{raiz\PYZus{}n}\PY{p}{:}
                \PY{k}{break}
                
            \PY{c+c1}{\PYZsh{} Si encontramos un factor primo, sabemos que n no es primo}
            \PY{k}{if} \PY{n}{n} \PY{o}{\PYZpc{}} \PY{n}{primo} \PY{o}{==} \PY{l+m+mi}{0}\PY{p}{:}
                \PY{n}{es\PYZus{}primo} \PY{o}{=} \PY{k+kc}{False}
                \PY{k}{break}  \PY{c+c1}{\PYZsh{} Dejamos de revisar inmediatamente}
        
        \PY{c+c1}{\PYZsh{} Si no encontramos ningún factor primo ≤ √n, entonces n es primo}
        \PY{k}{if} \PY{n}{es\PYZus{}primo}\PY{p}{:}
            \PY{n}{primos}\PY{o}{.}\PY{n}{append}\PY{p}{(}\PY{n}{n}\PY{p}{)}
    
    \PY{k}{return} \PY{n}{primos}
\end{Verbatim}
\end{tcolorbox}

    \begin{tcolorbox}[breakable, size=fbox, boxrule=1pt, pad at break*=1mm,colback=cellbackground, colframe=cellborder]
\prompt{In}{incolor}{ }{\boxspacing}
\begin{Verbatim}[commandchars=\\\{\}]
\PY{k}{def} \PY{n+nf}{main}\PY{p}{(}\PY{p}{)}\PY{p}{:}
    \PY{c+c1}{\PYZsh{} Encontrar todos los primos hasta 10,000}
    \PY{n}{primos} \PY{o}{=} \PY{n}{encontrar\PYZus{}primos\PYZus{}hasta}\PY{p}{(}\PY{l+m+mi}{10000}\PY{p}{)}
    \PY{n+nb}{print}\PY{p}{(}\PY{l+s+sa}{f}\PY{l+s+s2}{\PYZdq{}}\PY{l+s+s2}{Se encontraron }\PY{l+s+si}{\PYZob{}}\PY{n+nb}{len}\PY{p}{(}\PY{n}{primos}\PY{p}{)}\PY{l+s+si}{\PYZcb{}}\PY{l+s+s2}{ números primos hasta 10,000}\PY{l+s+s2}{\PYZdq{}}\PY{p}{)}
    \PY{n+nb}{print}\PY{p}{(}\PY{n}{primos}\PY{p}{)}

\PY{k}{if} \PY{n+nv+vm}{\PYZus{}\PYZus{}name\PYZus{}\PYZus{}} \PY{o}{==} \PY{l+s+s2}{\PYZdq{}}\PY{l+s+s2}{\PYZus{}\PYZus{}main\PYZus{}\PYZus{}}\PY{l+s+s2}{\PYZdq{}}\PY{p}{:}
    \PY{n}{main}\PY{p}{(}\PY{p}{)}
\end{Verbatim}
\end{tcolorbox}

    \begin{tcolorbox}[breakable, size=fbox, boxrule=1pt, pad at break*=1mm,colback=cellbackground, colframe=cellborder]
\prompt{In}{incolor}{ }{\boxspacing}
\begin{Verbatim}[commandchars=\\\{\}]
\PY{k}{def} \PY{n+nf}{encontrar\PYZus{}primos\PYZus{}con\PYZus{}impresion}\PY{p}{(}\PY{n}{limite}\PY{p}{)}\PY{p}{:}
    \PY{l+s+sd}{\PYZdq{}\PYZdq{}\PYZdq{}}
\PY{l+s+sd}{    Encuentra primos e imprime cada uno a medida que se encuentra}
\PY{l+s+sd}{    \PYZdq{}\PYZdq{}\PYZdq{}}
    \PY{n}{primos} \PY{o}{=} \PY{p}{[}\PY{l+m+mi}{2}\PY{p}{]}
    \PY{n+nb}{print}\PY{p}{(}\PY{l+s+sa}{f}\PY{l+s+s2}{\PYZdq{}}\PY{l+s+s2}{Primo encontrado: 2}\PY{l+s+s2}{\PYZdq{}}\PY{p}{)}
    
    \PY{k}{for} \PY{n}{n} \PY{o+ow}{in} \PY{n+nb}{range}\PY{p}{(}\PY{l+m+mi}{3}\PY{p}{,} \PY{n}{limite} \PY{o}{+} \PY{l+m+mi}{1}\PY{p}{)}\PY{p}{:}
        \PY{n}{raiz\PYZus{}n} \PY{o}{=} \PY{n}{math}\PY{o}{.}\PY{n}{isqrt}\PY{p}{(}\PY{n}{n}\PY{p}{)}
        \PY{n}{es\PYZus{}primo} \PY{o}{=} \PY{k+kc}{True}
        
        \PY{k}{for} \PY{n}{primo} \PY{o+ow}{in} \PY{n}{primos}\PY{p}{:}
            \PY{k}{if} \PY{n}{primo} \PY{o}{\PYZgt{}} \PY{n}{raiz\PYZus{}n}\PY{p}{:}
                \PY{k}{break}
            \PY{k}{if} \PY{n}{n} \PY{o}{\PYZpc{}} \PY{n}{primo} \PY{o}{==} \PY{l+m+mi}{0}\PY{p}{:}
                \PY{n}{es\PYZus{}primo} \PY{o}{=} \PY{k+kc}{False}
                \PY{k}{break}
        
        \PY{k}{if} \PY{n}{es\PYZus{}primo}\PY{p}{:}
            \PY{n}{primos}\PY{o}{.}\PY{n}{append}\PY{p}{(}\PY{n}{n}\PY{p}{)}
            \PY{n+nb}{print}\PY{p}{(}\PY{l+s+sa}{f}\PY{l+s+s2}{\PYZdq{}}\PY{l+s+s2}{Primo encontrado: }\PY{l+s+si}{\PYZob{}}\PY{n}{n}\PY{l+s+si}{\PYZcb{}}\PY{l+s+s2}{\PYZdq{}}\PY{p}{)}
    
    \PY{k}{return} \PY{n}{primos}

\PY{c+c1}{\PYZsh{} Ejecutar esta versión}
\PY{n+nb}{print}\PY{p}{(}\PY{l+s+s2}{\PYZdq{}}\PY{l+s+s2}{Buscando primos con impresión individual...}\PY{l+s+s2}{\PYZdq{}}\PY{p}{)}
\PY{n}{primos\PYZus{}lista} \PY{o}{=} \PY{n}{encontrar\PYZus{}primos\PYZus{}con\PYZus{}impresion}\PY{p}{(}\PY{l+m+mi}{10000}\PY{p}{)}
\PY{n+nb}{print}\PY{p}{(}\PY{l+s+sa}{f}\PY{l+s+s2}{\PYZdq{}}\PY{l+s+se}{\PYZbs{}n}\PY{l+s+s2}{Total: }\PY{l+s+si}{\PYZob{}}\PY{n+nb}{len}\PY{p}{(}\PY{n}{primos\PYZus{}lista}\PY{p}{)}\PY{l+s+si}{\PYZcb{}}\PY{l+s+s2}{ primos encontrados}\PY{l+s+s2}{\PYZdq{}}\PY{p}{)}
\end{Verbatim}
\end{tcolorbox}

    \begin{tcolorbox}[breakable, size=fbox, boxrule=1pt, pad at break*=1mm,colback=cellbackground, colframe=cellborder]
\prompt{In}{incolor}{ }{\boxspacing}
\begin{Verbatim}[commandchars=\\\{\}]
\PY{k}{def} \PY{n+nf}{filtrar\PYZus{}primos\PYZus{}con\PYZus{}2}\PY{p}{(}\PY{n}{primos}\PY{p}{)}\PY{p}{:}
    \PY{l+s+sd}{\PYZdq{}\PYZdq{}\PYZdq{}}
\PY{l+s+sd}{    Filtra solo los números primos que comienzan con el dígito 2}
\PY{l+s+sd}{    \PYZdq{}\PYZdq{}\PYZdq{}}
    \PY{n}{primos\PYZus{}con\PYZus{}2} \PY{o}{=} \PY{p}{[}\PY{p}{]}
    \PY{k}{for} \PY{n}{primo} \PY{o+ow}{in} \PY{n}{primos}\PY{p}{:}
        \PY{k}{if} \PY{n+nb}{str}\PY{p}{(}\PY{n}{primo}\PY{p}{)}\PY{o}{.}\PY{n}{startswith}\PY{p}{(}\PY{l+s+s1}{\PYZsq{}}\PY{l+s+s1}{2}\PY{l+s+s1}{\PYZsq{}}\PY{p}{)}\PY{p}{:}
            \PY{n}{primos\PYZus{}con\PYZus{}2}\PY{o}{.}\PY{n}{append}\PY{p}{(}\PY{n}{primo}\PY{p}{)}
    \PY{k}{return} \PY{n}{primos\PYZus{}con\PYZus{}2}

\PY{c+c1}{\PYZsh{} Ejemplo de uso}
\PY{n}{todos\PYZus{}primos} \PY{o}{=} \PY{n}{encontrar\PYZus{}primos\PYZus{}hasta}\PY{p}{(}\PY{l+m+mi}{10000}\PY{p}{)}
\PY{n}{primos\PYZus{}con\PYZus{}2} \PY{o}{=} \PY{n}{filtrar\PYZus{}primos\PYZus{}con\PYZus{}2}\PY{p}{(}\PY{n}{todos\PYZus{}primos}\PY{p}{)}
\PY{n+nb}{print}\PY{p}{(}\PY{l+s+sa}{f}\PY{l+s+s2}{\PYZdq{}}\PY{l+s+s2}{Primos que comienzan con 2: }\PY{l+s+si}{\PYZob{}}\PY{n+nb}{len}\PY{p}{(}\PY{n}{primos\PYZus{}con\PYZus{}2}\PY{p}{)}\PY{l+s+si}{\PYZcb{}}\PY{l+s+s2}{\PYZdq{}}\PY{p}{)}
\PY{n+nb}{print}\PY{p}{(}\PY{n}{primos\PYZus{}con\PYZus{}2}\PY{p}{)}
\end{Verbatim}
\end{tcolorbox}

    \begin{tcolorbox}[breakable, size=fbox, boxrule=1pt, pad at break*=1mm,colback=cellbackground, colframe=cellborder]
\prompt{In}{incolor}{ }{\boxspacing}
\begin{Verbatim}[commandchars=\\\{\}]
\PY{k}{def} \PY{n+nf}{guardar\PYZus{}primos\PYZus{}con\PYZus{}2}\PY{p}{(}\PY{n}{primos}\PY{p}{,} \PY{n}{nombre\PYZus{}archivo}\PY{o}{=}\PY{l+s+s1}{\PYZsq{}}\PY{l+s+s1}{primos\PYZus{}con\PYZus{}2.txt}\PY{l+s+s1}{\PYZsq{}}\PY{p}{)}\PY{p}{:}
    \PY{l+s+sd}{\PYZdq{}\PYZdq{}\PYZdq{}}
\PY{l+s+sd}{    Guarda los primos que comienzan con 2 en un archivo de texto}
\PY{l+s+sd}{    \PYZdq{}\PYZdq{}\PYZdq{}}
    \PY{n}{primos\PYZus{}con\PYZus{}2} \PY{o}{=} \PY{n}{filtrar\PYZus{}primos\PYZus{}con\PYZus{}2}\PY{p}{(}\PY{n}{primos}\PY{p}{)}
    
    \PY{k}{with} \PY{n+nb}{open}\PY{p}{(}\PY{n}{nombre\PYZus{}archivo}\PY{p}{,} \PY{l+s+s1}{\PYZsq{}}\PY{l+s+s1}{w}\PY{l+s+s1}{\PYZsq{}}\PY{p}{)} \PY{k}{as} \PY{n}{archivo}\PY{p}{:}
        \PY{k}{for} \PY{n}{primo} \PY{o+ow}{in} \PY{n}{primos\PYZus{}con\PYZus{}2}\PY{p}{:}
            \PY{n}{archivo}\PY{o}{.}\PY{n}{write}\PY{p}{(}\PY{n+nb}{str}\PY{p}{(}\PY{n}{primo}\PY{p}{)} \PY{o}{+} \PY{l+s+s1}{\PYZsq{}}\PY{l+s+se}{\PYZbs{}n}\PY{l+s+s1}{\PYZsq{}}\PY{p}{)}
    
    \PY{n+nb}{print}\PY{p}{(}\PY{l+s+sa}{f}\PY{l+s+s2}{\PYZdq{}}\PY{l+s+s2}{¡Archivo }\PY{l+s+s2}{\PYZsq{}}\PY{l+s+si}{\PYZob{}}\PY{n}{nombre\PYZus{}archivo}\PY{l+s+si}{\PYZcb{}}\PY{l+s+s2}{\PYZsq{}}\PY{l+s+s2}{ creado exitosamente!}\PY{l+s+s2}{\PYZdq{}}\PY{p}{)}

\PY{c+c1}{\PYZsh{} Ejecutar}
\PY{n}{todos\PYZus{}primos} \PY{o}{=} \PY{n}{encontrar\PYZus{}primos\PYZus{}hasta}\PY{p}{(}\PY{l+m+mi}{10000}\PY{p}{)}
\PY{n}{guardar\PYZus{}primos\PYZus{}con\PYZus{}2}\PY{p}{(}\PY{n}{todos\PYZus{}primos}\PY{p}{)}
\end{Verbatim}
\end{tcolorbox}

    \begin{tcolorbox}[breakable, size=fbox, boxrule=1pt, pad at break*=1mm,colback=cellbackground, colframe=cellborder]
\prompt{In}{incolor}{ }{\boxspacing}
\begin{Verbatim}[commandchars=\\\{\}]

\end{Verbatim}
\end{tcolorbox}


    % Add a bibliography block to the postdoc
    
    
    
\end{document}
